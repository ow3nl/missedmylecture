\documentclass[12pt,a4paper]{article}
\usepackage{amsmath}
\usepackage{amssymb}
\usepackage{geometry}

\geometry{
    left=1in,
    right=1in,
    top=1in,
    bottom=1in
}

\begin{document}

\section{Inverse Matrices}

An \textit{inverse matrix} A is invertible if there is a matrix A' such that AA' = A'A = I.

\subsection{Examples}

\begin{align*}
A &= \begin{bmatrix} 2 & 3 \\ 1 & 2 \end{bmatrix}, \quad B = \begin{bmatrix} 2 & -3 \\ -1 & 2 \end{bmatrix} \text{ are inverses} \\
\text{Since } AB &= \begin{bmatrix} 1 & 0 \\ 0 & 1 \end{bmatrix} = BA \text{ we can write B = A}^{-1}
\end{align*}

Not all matrices have inverses:
\begin{align*}
C &= \begin{bmatrix} 0 & 0 \\ 1 & 0 \end{bmatrix} \text{ has no inverse} \\
\text{Since } AC &= \begin{bmatrix} 0 & 0 \\ 0 & 0 \end{bmatrix} \neq I
\end{align*}

\subsection{Connection to Linear Transformations}

Let T be a linear transformation with associated matrix A (in T:A x V→W).
The inverse of T is a linear transformation with associated matrix A' where A' = A\textsuperscript{-1}.

\section{2x2 Inverse Formula}

Let $A = \begin{bmatrix} a & b \\ c & d \end{bmatrix}$, then if ad-bc≠0:

\[
A^{-1} = \frac{1}{ad-bc} \begin{bmatrix} d & -b \\ -c & a \end{bmatrix}
\]

\textit{Check:} Show AA' = I and A'A = I

There is a general nxn inverse formula (involving the determinant and adjugate matrix) but it's complicated to compute (we'll see this later).

\subsection{Using Gauss-Jordan to Find the Inverse}

Suppose A has inverse A'.
Then AA' = I
So A[I] = [A']

To find A', we must solve the system Ax = e\textsubscript{i} for each column.

Instead of solving [Ax\textsubscript{1}] = [e\textsubscript{1}], [Ax\textsubscript{2}] = [e\textsubscript{2}], ... individually, we can solve them all at once with Gauss-Jordan on the matrix [A|I].

If A doesn't have n pivots (one in each row), then A' doesn't exist.
If A has n pivots, then the algorithm produces A'.

\subsection{Summary: Procedure to Find Inverse (if it exists)}

\begin{itemize}
    \item If A doesn't have n pivots, then A' doesn't exist
    \item If A has n pivots, then the algorithm produces A'
\end{itemize}

\section{Example}

Let $A = \begin{bmatrix} 2 & 1 & 3 \\ 0 & 1 & 4 \\ 1 & 0 & 1 \end{bmatrix}$. Find A\textsuperscript{-1}, if it exists.

\begin{align*}
\begin{bmatrix}
2 & 1 & 3 & | & 1 & 0 & 0 \\
0 & 1 & 4 & | & 0 & 1 & 0 \\
1 & 0 & 1 & | & 0 & 0 & 1
\end{bmatrix}
&\xrightarrow{\text{R}_3 - \frac{1}{2}\text{R}_1}
\begin{bmatrix}
2 & 1 & 3 & | & 1 & 0 & 0 \\
0 & 1 & 4 & | & 0 & 1 & 0 \\
0 & -\frac{1}{2} & -\frac{1}{2} & | & -\frac{1}{2} & 0 & 1
\end{bmatrix} \\
&\xrightarrow{\text{R}_3 + \frac{1}{2}\text{R}_2}
\begin{bmatrix}
2 & 1 & 3 & | & 1 & 0 & 0 \\
0 & 1 & 4 & | & 0 & 1 & 0 \\
0 & 0 & \frac{3}{2} & | & -\frac{1}{2} & \frac{1}{2} & 1
\end{bmatrix} \\
&\xrightarrow{\text{R}_3 \cdot \frac{2}{3}}
\begin{bmatrix}
2 & 1 & 3 & | & 1 & 0 & 0 \\
0 & 1 & 4 & | & 0 & 1 & 0 \\
0 & 0 & 1 & | & -\frac{1}{3} & \frac{1}{3} & \frac{2}{3}
\end{bmatrix} \\
&\xrightarrow{\text{R}_2 - 4\text{R}_3}
\begin{bmatrix}
2 & 1 & 3 & | & 1 & 0 & 0 \\
0 & 1 & 0 & | & \frac{4}{3} & -\frac{1}{3} & -\frac{8}{3} \\
0 & 0 & 1 & | & -\frac{1}{3} & \frac{1}{3} & \frac{2}{3}
\end{bmatrix} \\
&\xrightarrow{\text{R}_1 - 3\text{R}_3}
\begin{bmatrix}
2 & 1 & 0 & | & 2 & -1 & -2 \\
0 & 1 & 0 & | & \frac{4}{3} & -\frac{1}{3} & -\frac{8}{3} \\
0 & 0 & 1 & | & -\frac{1}{3} & \frac{1}{3} & \frac{2}{3}
\end{bmatrix} \\
&\xrightarrow{\text{R}_1 - \text{R}_2}
\begin{bmatrix}
2 & 0 & 0 & | & \frac{2}{3} & -\frac{2}{3} & \frac{2}{3} \\
0 & 1 & 0 & | & \frac{4}{3} & -\frac{1}{3} & -\frac{8}{3} \\
0 & 0 & 1 & | & -\frac{1}{3} & \frac{1}{3} & \frac{2}{3}
\end{bmatrix} \\
&\xrightarrow{\text{R}_1 \cdot \frac{1}{2}}
\begin{bmatrix}
1 & 0 & 0 & | & \frac{1}{3} & -\frac{1}{3} & \frac{1}{3} \\
0 & 1 & 0 & | & \frac{4}{3} & -\frac{1}{3} & -\frac{8}{3} 