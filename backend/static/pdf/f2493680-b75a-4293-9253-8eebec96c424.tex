\documentclass[12pt,a4paper]{article}
\usepackage{amsmath}
\usepackage{amssymb}
\usepackage{geometry}

\geometry{margin=1in}

\begin{document}

\section{Inverse Matrices}

An \textit{inverse matrix} is invertible if there is a matrix $A^{-1}$ such that $AA^{-1} = A^{-1}A = I$.

\subsection{Examples}

\begin{itemize}
    \item $A = \begin{bmatrix} 3 & 1 \\ 2 & 1 \end{bmatrix}$, $B = \begin{bmatrix} 1 & -1 \\ -2 & 3 \end{bmatrix}$ are inverses
    \item Since $AB = \begin{bmatrix} 1 & 0 \\ 0 & 1 \end{bmatrix}$, we can write $B = A^{-1}$
    \item Not all matrices have inverses. $A = \begin{bmatrix} 0 & 0 \\ 0 & 0 \end{bmatrix}$ has no inverse
    \item Similarly, $AB = \begin{bmatrix} 0 & 0 \\ 0 & 0 \end{bmatrix} \neq I$
\end{itemize}

\subsection{Connection to Linear Transformations}

$Ax = b$ has a unique solution if and only if $A$ is invertible. The inverse transformation takes $b$ back to $x$.

\section{2x2 Inverse Formula}

Let $A = \begin{bmatrix} a & b \\ c & d \end{bmatrix}$, then if $ad-bc \neq 0$:

\[
A^{-1} = \frac{1}{ad-bc} \begin{bmatrix} d & -b \\ -c & a \end{bmatrix}
\]

Note: $\det A = ad-bc$, and $AA^{-1} = I$ and $A^{-1}A = I$

This is a special case of a more general formula (involving the determinant and adjugate matrix, which are complicated to compute but follow this idea).

\subsection{Using Gauss-Jordan to Find the Inverse}

Suppose $A$ has inverse $A^{-1}$. Then:

$AA^{-1} = I$
$[A|I] \to [I|A^{-1}]$

So to find $A^{-1}$, we must solve the system $Ax = e_i$ for each $i$.

Instead of solving $[A|e_1], [A|e_2], \ldots, [A|e_n]$ individually, we can streamline the process and solve $[A|I]$ at once.

If $A$ doesn't have $n$ pivots (one in each row), then $A^{-1}$ does not exist.

If $A$ has $n$ pivots, then the algorithm produces $A^{-1}$.

\subsection{Summary for Gauss-Jordan Method}
\begin{itemize}
    \item If $A$ doesn't have $n$ pivots, then $A^{-1}$ doesn't exist
    \item If $A$ has $n$ pivots, then we get $A^{-1}$
\end{itemize}

\section{Example}

Find $A^{-1}$ if it exists:

\[
A = \begin{bmatrix} 
2 & 1 & 3 \\
2 & 6 & 8 \\
6 & 8 & 18
\end{bmatrix}
\]

\[
[A|I] = \begin{bmatrix} 
2 & 1 & 3 & | & 1 & 0 & 0 \\
2 & 6 & 8 & | & 0 & 1 & 0 \\
6 & 8 & 18 & | & 0 & 0 & 1
\end{bmatrix}
\]

(Steps of row operations omitted for brevity)

\[
[I|A^{-1}] = \begin{bmatrix} 
1 & 0 & 0 & | & 3 & -3 & 1 \\
0 & 1 & 0 & | & -5 & 4 & -1 \\
0 & 0 & 1 & | & 1 & -1 & 0
\end{bmatrix}
\]

Therefore:

\[
A^{-1} = \begin{bmatrix} 
3 & -3 & 1 \\
-5 & 4 & -1 \\
1 & -1 & 0
\end{bmatrix}
\]

(Check: verify $AA^{-1} = I$)

\section{Using Inverses to Solve Systems}

If $A$ exists, we can solve $Ax = b$ by multiplying both sides by $A^{-1}$:

\begin{align*}
A^{-1}Ax &= A^{-1}b \\
Ix &= A^{-1}b \\
x &= A^{-1}b
\end{align*}

Note: $A^{-1}b$ has solution $x$ if $b$ (whenever some $A$ has $n$ pivots), this solution is unique.

\end{document}