\documentclass[12pt,a4paper]{article}
\usepackage[margin=1in]{geometry}
\usepackage{amsmath}
\usepackage{amssymb}

\begin{document}

\section{Sets}

\subsection{Definitions}

\begin{align*}
\mathbb{N} &= \{0, 1, 2, \ldots\} &\text{natural numbers} \\
\mathbb{Z} &= \{\ldots, -1, 0, 1, \ldots\} &\text{integers} \\
\mathbb{Q} &= \{\frac{a}{b} : a, b \in \mathbb{Z}, b \neq 0\} &\text{rationals} \\
\mathbb{R} &= \text{real numbers} \\
\mathbb{C} &= \{a + bi : a, b \in \mathbb{R}\} &\text{complex numbers}
\end{align*}

$\mathbb{R}^n = \{(x_1, \ldots, x_n) : x_i \in \mathbb{R}\}$ \textit{n-dimensional space}

$\mathbb{R}^* = \{(x_1, x_2, \ldots) : x_i \in \mathbb{R}\}$ \textit{infinite dimensional space}

\subsection{Set Operations}

$A \cup B = \{x : x \in A \text{ or } x \in B\}$ \textit{union}

$A \cap B = \{x : x \in A \text{ and } x \in B\}$ \textit{intersection}

$A \setminus B = \{x \in A : x \notin B\}$ \textit{set difference}

$A \triangle B = (A \setminus B) \cup (B \setminus A)$ \textit{symmetric difference}

\subsection{Venn Diagrams}

[Venn diagrams would be inserted here]

\section{Functions}

Let $(f_n)_{n \in \mathbb{N}}$ be an indexed family of functions (where $n \in \mathbb{N}$).

\begin{align*}
\text{Ex } (1) \quad f &: \mathbb{N} \to \mathbb{N} \\
A_n &= \{k, n+1\} \\
&\text{definition}
\end{align*}

\begin{align*}
(2) \quad f &: \mathbb{R}^* \to \{x \in \mathbb{R} : x \geq 0\} \\
A_n &= (x_n, \infty)
\end{align*}

\begin{center}
$\bigcup_{n \in \mathbb{N}} A_n = \{x : x \in A_n \text{ for some } n \in \mathbb{N}\}$

$\bigcap_{n \in \mathbb{N}} A_n = \{x : x \in A_n \text{ for every } n \in \mathbb{N}\}$
\end{center}

These are called \textit{union} and \textit{intersection} of a family of sets $(A_n)_{n \in \mathbb{N}}$.

Let $A, B$ be sets $(A \neq B)$. A map $f : A \to B$ is called a \textit{function} or \textit{mapping} or \textit{transformation}.

If $A \subseteq B$ and $f : A \to B$, then $f$ is called an \textit{embedding} of $A$ into $B$ if $f$ is injective.

Note: $f$ maps $A$ onto $B$ $\Leftrightarrow$ $f$ is surjective

If $f : A \to B$ is both injective and surjective, then $f$ is called \textit{bijective} (or \textit{one-to-one}).

$f^{-1}$ is called the \textit{inverse} (if it exists) of $f$.

If $A' \subseteq A$, then the map $f' : A' \to B$ defined by $f'(x) = f(x)$ is called the \textit{restriction} of $f$ to $A'$ (denoted by $f|_{A'}$).

\textbf{Ex} (1) $f : \mathbb{R} \to \mathbb{R}$, $x \mapsto x^3$

(2) $f : \mathbb{R} \setminus \{0\} \to \mathbb{R}$, $x \mapsto \frac{1}{x}$

(3) $g : \mathbb{R} \to \mathbb{R}$, $x \mapsto e^x$

(4) $f : \mathbb{C} \to \mathbb{C}$, $z \mapsto z^2$

(5) $f : \mathbb{R}^2 \to \mathbb{R}$, $(x,y) \mapsto x+y$

(6) $D : C^1(\mathbb{R}) \to C(\mathbb{R})$, $f \mapsto f'$ \textit{derivative}

\end{document}