\documentclass[12pt,a4paper]{article}
\usepackage[margin=1in]{geometry}
\usepackage{amsmath,amssymb}

\begin{document}

\section{Set Theory}

\subsection{Basic Definitions}

\begin{align*}
\mathbb{N} &= \{0,1,2,\ldots\} & \text{\textit{natural numbers}} \\
\mathbb{Z} &= \{\ldots,-1,0,1,\ldots\} & \text{\textit{integers}} \\
\mathbb{Q} &= \{\frac{a}{b} : a,b \in \mathbb{Z}, b \neq 0\} & \text{\textit{rationals}} \\
\mathbb{R} &= \text{\textit{real numbers}} \\
\mathbb{C} &= \{a + bi : a,b \in \mathbb{R}\} & \text{\textit{complex numbers}}
\end{align*}

\subsection{Set Operations}

\begin{itemize}
\item $\mathbb{R}^n = \{(x_1,\ldots,x_n) : x_i \in \mathbb{R}\}$ \textit{n-dimensional space}
\item $\mathbb{R}^\infty = \{(x_1,x_2,\ldots) : x_i \in \mathbb{R}\}$ \textit{space of infinite sequences of reals}
\end{itemize}

\subsection{Set Relations}

\begin{align*}
A \subset B &: \text{\textit{A is a subset of B}} \\
A \cap B &: \text{\textit{intersection}} \\
A \cup B &: \text{\textit{union}}
\end{align*}

\begin{center}
[Venn diagrams for $A \cap B$ and $A \cup B$]
\end{center}

\begin{align*}
A \setminus B &= \{x \in A : x \notin B\} \\
A \triangle B &= (A \cup B) \setminus (A \cap B) \\
&= \text{\textit{symmetric difference}}
\end{align*}

\section{Intervals}

Let $a,b \in \mathbb{R}$ with $a < b$. Then:

\begin{align*}
(a,b) &= \{x \in \mathbb{R} : a < x < b\} \\
[a,b] &= \{x \in \mathbb{R} : a \leq x \leq b\}
\end{align*}

\section{Functions}

\begin{align*}
f: X \to Y &= \{(x,y) : x \in X, y \in Y\} \\
A_f &= \{x \in X : (x,y) \in f \text{ for some } y \in Y\} \\
&= \text{\textit{domain}} \\
\bigcup_{x \in X} f(x) &= \{y : y = f(x) \text{ for some } x \in X\} \\
&= \text{\textit{range}}
\end{align*}

\textit{These are generally:}
\begin{itemize}
\item $A_f = \text{domain} = \text{preimage of f}$
\item $f(A) = \text{image of A under f}$
\end{itemize}

\textit{A map} $f: A \to B$ \textit{is called:}
\begin{itemize}
\item \textit{injective} if $\forall x_1, x_2 \in A$, $f(x_1) = f(x_2) \Rightarrow x_1 = x_2$
\item \textit{surjective} if $\forall y \in B$, $\exists x \in A$ such that $f(x) = y$
\item \textit{bijective} if $f$ is both injective and surjective
\end{itemize}

\textit{Note:} $f$ \textit{injective} $\Leftrightarrow$ $f^{-1}$ \textit{exists}

$f^{-1}$ \textit{is called the inverse of} $f$.

$\Delta$ \textit{is called the diagonal} (by definition) of $X$.

\section{Composition of Functions}

If $f: A \to B$ and $g: B \to C$, then the composition of $f$ and $g$ (denoted by $g \circ f$) is:

\begin{center}
$g \circ f: A \to C$
\end{center}

\textit{This is called the composition of} $f$ \textit{and} $g$.

\subsection{Examples}
\begin{enumerate}
\item $f: \mathbb{R} \to \mathbb{R}$, $x \mapsto \sin x$
\item $f: \mathbb{R}[X] \to \mathbb{R}$, $p \mapsto p(0)$
\item $g: \mathbb{R} \to \mathbb{R}$, $x \mapsto e^x$
\item $f: \mathbb{R} \to \mathbb{R}$, $x \mapsto \frac{1}{x}$
\item $f: V \to V^*$, $v \mapsto (v,-)$
\end{enumerate}

\end{document}