\documentclass[12pt,a4paper]{article}
\usepackage[margin=1in]{geometry}
\usepackage{amsmath}

\begin{document}

\section{Inverse Matrices}

An non-singular matrix A is invertible if there is a matrix A' such that AA' = A'A = I.

\subsection{Examples}

\begin{align*}
A = \begin{bmatrix} 2 & 1 \\ 1 & 3 \end{bmatrix}, \quad B = \begin{bmatrix} 3 & -1 \\ -1 & 2 \end{bmatrix} \text{ are inverses}
\end{align*}

Since AB = BA = I, we can write B = A'.

Not all matrices have inverses.

\begin{align*}
A = \begin{bmatrix} 0 & 0 \\ 0 & 1 \end{bmatrix} \text{ has no inverse}
\end{align*}

Since AB = \begin{bmatrix} 0 & 0 \\ 0 & 1 \end{bmatrix} $\neq$ I.

\subsection{Connection to Linear Transformations}

A' : T' $\circ$ T = I (i.e., T' $\circ$ T is the identity transformation)

The inverse of T is a linear transformation with associated matrix A'.

\section{2x2 Inverse Formula}

Let $A = \begin{bmatrix} a & b \\ c & d \end{bmatrix}$. Then if ad-bc $\neq$ 0,

\begin{align*}
A^{-1} = \frac{1}{ad-bc} \begin{bmatrix} d & -b \\ -c & a \end{bmatrix}
\end{align*}

\textit{Check}: Show AA' = I and A'A = I

There is a general nxn inverse formula (involving the determinant and adjoint matrix) but it's complicated to compute (we'll see this later).

For now, we'll use Gauss-Jordan to find the inverse (if it exists).

Suppose A has inverse A'.
Then AA' = I
$\implies [A|I] \sim [I|A']$ (by row operations)

So to find A', we must solve the system Ax = e$_i$ for each i.

Instead of solving [Ax$_1$ = e$_1$], [Ax$_2$ = e$_2$], ..., [Ax$_n$ = e$_n$] individually, we can streamline the process by solving one augmented matrix:

$[A|e_1, e_2, ..., e_n] = [A|I]$

If A doesn't have n pivots (one in each row), then A' doesn't exist.
If A has n pivots, then the algorithm produces A'.

\section{Summary: Algorithm to find Inverse (if it exists)}

\begin{enumerate}
    \item If A doesn't have n pivots, then A' doesn't exist
    \item If A has n pivots, then the algorithm produces A'
\end{enumerate}

\section{Example}

Let $A = \begin{bmatrix} 2 & 1 & 0 \\ 1 & 2 & 1 \\ 0 & 1 & 2 \end{bmatrix}$. Find A', if it exists.

\begin{align*}
\begin{bmatrix}
2 & 1 & 0 & | & 1 & 0 & 0 \\
1 & 2 & 1 & | & 0 & 1 & 0 \\
0 & 1 & 2 & | & 0 & 0 & 1
\end{bmatrix} \sim
\begin{bmatrix}
1 & 0 & 0 & | & \frac{3}{4} & -\frac{1}{4} & -\frac{1}{8} \\
0 & 1 & 0 & | & -\frac{1}{2} & \frac{1}{2} & -\frac{1}{4} \\
0 & 0 & 1 & | & \frac{1}{8} & -\frac{1}{4} & \frac{5}{8}
\end{bmatrix}
\end{align*}

$\therefore A^{-1} = \begin{bmatrix} \frac{3}{4} & -\frac{1}{4} & -\frac{1}{8} \\ -\frac{1}{2} & \frac{1}{2} & -\frac{1}{4} \\ \frac{1}{8} & -\frac{1}{4} & \frac{5}{8} \end{bmatrix}$ (Check: Verify AA' = I)

\section{Using Inverses to Solve Systems}

If A exists, we can solve Ax = b by multiplying by A':

\begin{align*}
A'(Ax) &= A'b \\
(A'A)x &= A'b \\
Ix &= A'b \\
x &= A'b
\end{align*}

Hence Ax = b has solution x = A'b. However, given A has n pivots (is invertible), this solution is unique.

\end{document}