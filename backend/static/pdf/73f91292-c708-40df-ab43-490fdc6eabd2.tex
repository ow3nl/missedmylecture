\documentclass[12pt,a4paper]{article}
\usepackage[margin=1in]{geometry}
\usepackage{amsmath,amssymb}

\begin{document}

\section{Set Theory}

\subsection{Sets and Notations}

\begin{itemize}
    \item $\mathbb{N} = \{0,1,2,\ldots\}$ - natural numbers
    \item $\mathbb{Z} = \{\ldots,-1,0,1,\ldots\}$ - integers
    \item $\mathbb{Q} = \{\frac{p}{q} : p,q \in \mathbb{Z}, q \neq 0\}$ - rationals
    \item $\mathbb{R}$ - real numbers
    \item $\mathbb{C} = \{a+bi : a,b \in \mathbb{R}\}$ - complex numbers
\end{itemize}

\subsection{Set Operations}

\begin{itemize}
    \item $\mathbb{R}^n = \{(x_1,\ldots,x_n) : x_i \in \mathbb{R}\}$ - $n$-dimensional space
    \item $\mathbb{R}^\infty = \{(x_1,x_2,\ldots) : x_i \in \mathbb{R}\}$ - space of infinite sequences of reals
\end{itemize}

\subsection{Set Relations}

\begin{itemize}
    \item $A \cup B = \{x : x \in A \text{ or } x \in B\}$ - union
    \item $A \cap B = \{x : x \in A \text{ and } x \in B\}$ - intersection
    \item $A \setminus B = \{x \in A : x \notin B\}$ - difference
    \item $A \triangle B = (A \setminus B) \cup (B \setminus A)$ - symmetric difference
\end{itemize}

\subsection{Properties}

Let $\{A_\alpha\}_{\alpha \in I}$ be an indexed family of sets. Then:

\begin{enumerate}
    \item $I \subseteq \mathbb{N}$
    \item $A_\alpha = \{x,y,z\}$
    \item $I = \mathbb{R}^+$, $A_\alpha = [0,\alpha]$
\end{enumerate}

Define:
\begin{align*}
    \bigcup_{\alpha \in I} A_\alpha &= \{x : x \in A_\alpha \text{ for some } \alpha \in I\} \\
    \bigcap_{\alpha \in I} A_\alpha &= \{x : x \in A_\alpha \text{ for every } \alpha \in I\}
\end{align*}

\subsection{Functions}

A \textit{function} is a mapping or transformation.

Let $A, B$ be sets. $f:A \to B$ means:
\begin{itemize}
    \item $f$ assigns to each $a \in A$ a unique $b \in B$
    \item $a \mapsto b$ or $f(a) = b$
\end{itemize}

$f$ is \textit{surjective} if $\forall b \in B, \exists a \in A$ such that $f(a) = b$.

$f$ is \textit{injective} if $a_1 \neq a_2 \implies f(a_1) \neq f(a_2)$.

$f$ is \textit{bijective} if it is both surjective and injective.

\subsection{Inverse Function}

If $f : A \to B$ is bijective, then the \textit{inverse function} of $f$ (denoted by $f^{-1} : B \to A$) is defined by $f^{-1}(b) = a$.

Examples:
\begin{enumerate}
    \item $f : \mathbb{R} \to \mathbb{R}$, $f(x) = x^3$
    \item $f : \mathbb{R} \setminus \{0\} \to \mathbb{R} \setminus \{0\}$, $f(x) = \frac{1}{x}$
    \item $g : \mathbb{R} \to \mathbb{R}$, $x \mapsto x^2$
    \item $f : \mathbb{R} \to [0,\infty)$, $x \mapsto |x|$
    \item $f : \mathbb{R} \to (-1,1)$, $x \mapsto \tanh(x)$
\end{enumerate}

\end{document}