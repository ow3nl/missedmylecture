\documentclass[12pt,a4paper]{article}
\usepackage{amsmath}
\usepackage[margin=1in]{geometry}

\begin{document}

\section{Inverse Matrices}

An \textit{inverse matrix} is invertible if there is a matrix $A^{-1}$ such that $AA^{-1} = A^{-1}A = I$.

\subsection{Examples}

\begin{itemize}
    \item $A = \begin{bmatrix} 2 & 3 \\ 1 & 2 \end{bmatrix}$, $B = \begin{bmatrix} 2 & -3 \\ -1 & 2 \end{bmatrix}$ are inverses
    \item Since $AB = \begin{bmatrix} 1 & 0 \\ 0 & 1 \end{bmatrix}$, we can write $B = A^{-1}$
    \item Not all matrices have inverses. $A = \begin{bmatrix} 1 & 2 \\ 2 & 4 \end{bmatrix}$ has no inverse.
\end{itemize}

\subsection{Connection to Linear Transformations}

$A^{-1}$ is the inverse transformation with respect to matrix $A$ (or $T_A$ if $A$ is viewed as the linear transformation). The inverse of $T$ is a linear transformation whose associated matrix is $A^{-1}$.

\section{2x2 Inverse Formula}

Let $A = \begin{bmatrix} a & b \\ c & d \end{bmatrix}$, then if $ad-bc \neq 0$:

\[A^{-1} = \frac{1}{ad-bc} \begin{bmatrix} d & -b \\ -c & a \end{bmatrix}\]

This is a special non-unique formula (works only for 2x2). For larger matrices, it's more complicated to compute (we'll see this later).

\subsection{Using Cross-Section to Find the Inverse}

Suppose $A$ has inverse $A^{-1}$
Then $AA^{-1} = [A_1 A_2] \begin{bmatrix} | & | \\ x & y \\ | & | \end{bmatrix} = I_2$

So $A_1x = e_1$ and $A_2y = e_2$

To find $A^{-1}$, we must solve the system $Ax = e_i$ for each column vector of the identity matrix.

Instead of solving $[Ax], [Ay], \ldots, [Az]$ individually, we can streamline the process by solving one augmented matrix:

\[[A|e_1, e_2, \ldots]\]

If $A$ doesn't have $n$ pivots (one in each row), then $A^{-1}$ does not exist.

If $A$ has $n$ pivots, then the algorithm produces $A^{-1}$.

\subsection{Summary}
\begin{itemize}
    \item If $A$ doesn't have $n$ pivots, then $A^{-1}$ doesn't exist
    \item If $A$ has $n$ pivots, then we can find $A^{-1}$ using the RREF of $[A|I]$
\end{itemize}

\subsection{Example}

Find $A^{-1}$, if it exists:

\[A = \begin{bmatrix} 2 & 1 & 3 \\ 1 & 0 & 1 \\ 3 & 1 & 4 \end{bmatrix}\]

\[
\begin{bmatrix}
2 & 1 & 3 & | & 1 & 0 & 0 \\
1 & 0 & 1 & | & 0 & 1 & 0 \\
3 & 1 & 4 & | & 0 & 0 & 1
\end{bmatrix}
\sim
\begin{bmatrix}
1 & 0 & 0 & | & -1 & 1 & 0 \\
0 & 1 & 0 & | & 3 & -2 & -1 \\
0 & 0 & 1 & | & 1 & -1 & 1
\end{bmatrix}
\]

Therefore, 
\[A^{-1} = \begin{bmatrix} -1 & 1 & 0 \\ 3 & -2 & -1 \\ 1 & -1 & 1 \end{bmatrix} \quad \text{(Check: } AA^{-1} = I\text{)}\]

\section{Using Inverses to Solve Systems}

If $A$ exists, we can solve $Ax = b$ by multiplying $A^{-1}b$:

\[A^{-1}Ax = A^{-1}b\]
\[(A^{-1}A)x = A^{-1}b\]
\[Ix = A^{-1}b\]
\[x = A^{-1}b\]

Here $A^{-1}b$ is a solution to $Ax = b$. However, even if $A$ has $n$ pivots (is $n \times n$ invertible), this solution is unique.

\end{document}