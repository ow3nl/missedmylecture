\documentclass[12pt,a4paper]{article}
\usepackage[margin=1in]{geometry}
\usepackage{amsmath}
\usepackage{amssymb}

\begin{document}

\section{Sets and Definitions}

\subsection{Basic Sets}
$\mathbb{N} = \{0, 1, 2, \ldots\}$ - natural numbers \\
$\mathbb{Z} = \{\ldots, -1, 0, 1, \ldots\}$ - integers \\
$\mathbb{Q} = \{\frac{a}{b} : a, b \in \mathbb{Z}, b \neq 0\}$ - rationals \\
$\mathbb{R}$ - real numbers \\
$\mathbb{C} = \{a + bi : a, b \in \mathbb{R}\}$ - complex numbers

\subsection{Set Operations}
$\cup$ - union \\
$\cap$ - intersection \\
$\setminus$ - set difference \\
$\mathbb{R}^n = \{(x_1, \ldots, x_n) : x_i \in \mathbb{R}\}$ - $n$-dimensional space

$\mathbb{R}^\infty = \{(x_1, x_2, \ldots) : x_i \in \mathbb{R}\}$ - infinite-dimensional space of real sequences

\subsection{Set Relations}
$A \subset B$ - subset \\
$A \cap B = \{x : x \in A \text{ or } x \in B\}$ - union \\
$A \cap B = \{x : x \in A \text{ and } x \in B\}$ - intersection

\begin{center}
$A \setminus B = \{x \in A : x \notin B\}$
\end{center}

$A \triangle B = (A \setminus B) \cup (B \setminus A)$ - symmetric difference

\subsection{Examples}
Let $(a_n)_{n \in \mathbb{N}}$ be an infinite sequence (where $a_n \in \mathbb{R}$).

\begin{enumerate}
\item $I = \mathbb{N}$
\item $A_n = [n, \infty)$
\item $I = \mathbb{R}^+$, $A_x = (x, \infty)$
\end{enumerate}

\begin{center}
$\bigcup_{x \in I} A_x = \{y : y \in A_x \text{ for some } x \in I\}$

$\bigcap_{x \in I} A_x = \{y : y \in A_x \text{ for every } x \in I\}$
\end{center}

These are \textit{uncountable} unions/intersections.

\subsection{Functions and Mappings}
A map $f: A \to B$ is a rule that assigns to each $a \in A$ a unique $b \in B$.

$f$ is \textit{surjective} (or \textit{onto}) if $\forall b \in B, \exists a \in A$ such that $f(a) = b$.

$f$ is \textit{injective} (or \textit{one-to-one}) if $a_1 \neq a_2 \Rightarrow f(a_1) \neq f(a_2)$.

$f$ is \textit{bijective} if it is both surjective and injective.

\subsection{Inverse Functions}
If $f: A \to B$ is bijective, then the inverse function $f^{-1}: B \to A$ exists.

$(f^{-1} \circ f)(x) = x$ for all $x \in A$.

This is called the \textit{restriction} of $f$ to $A'$.

\subsection{Examples}
\begin{enumerate}
\item $f: \mathbb{R} \to \mathbb{R}, x \mapsto x^3$
\item $f: \mathbb{R} \setminus \{0\} \to \mathbb{R}, x \mapsto \frac{1}{x}$
\item $g: \mathbb{R} \to \mathbb{R}, x \mapsto e^x$
\item $f: \mathbb{R} \to (0, \infty), x \mapsto e^x$
\item $f: (-\frac{\pi}{2}, \frac{\pi}{2}) \to \mathbb{R}, x \mapsto \tan x$
\item $D: C(\mathbb{R}) \to C(\mathbb{R}), f \mapsto f'$ - derivative
\end{enumerate}

\end{document}