\documentclass[12pt,a4paper]{article}
\usepackage{amsmath}
\usepackage{amssymb}
\usepackage{geometry}

\geometry{
    top=2cm,
    bottom=2cm,
    left=2cm,
    right=2cm
}

\begin{document}

\section{Matrix Inverses}

An \textit{inverse} to a matrix $A$ is a matrix that $AA^{-1} = A^{-1}A = I$.

\subsection{Examples}

\begin{itemize}
    \item $A = \begin{bmatrix} 2 & 3 \\ 1 & 2 \end{bmatrix}, B = \begin{bmatrix} 2 & -3 \\ -1 & 2 \end{bmatrix}$ are inverses
    \item Since $AB = \begin{bmatrix} 1 & 0 \\ 0 & 1 \end{bmatrix}$ we can write $B = A^{-1}$
    \item Not all matrices have inverses. $A = \begin{bmatrix} 0 & 0 \\ 0 & 1 \end{bmatrix}$ has no inverse
    \item Some $AB = \begin{bmatrix} 1 & 0 \\ 0 & 1 \end{bmatrix} = I$, but $BA \neq I$
\end{itemize}

\subsection{Connection to Linear Transformations}

Let $T$ be a linear transformation with associated matrix $A$ (in TFVS). Then the inverse transformation has associated matrix $A^{-1}$.

\subsection{2x2 Inverse Formula}

Let $A = \begin{bmatrix} a & b \\ c & d \end{bmatrix}$, then if $ad-bc \neq 0$:

\[
A^{-1} = \frac{1}{ad-bc} \begin{bmatrix} d & -b \\ -c & a \end{bmatrix}
\]

This is a special case of a more general formula (involving the determinant and adjugate matrix, which are complicated to compute but follow this form).

\subsection{Using Gauss-Jordan to Find the Inverse}

Suppose $A$ has inverse $A^{-1}$. Then:

\[
AA^{-1} = I \implies [A|I] \sim [I|A^{-1}]
\]

So $A^{-1} = \begin{bmatrix} 2 & -3 \\ -1 & 2 \end{bmatrix}$ as found before.

To find $A^{-1}$, we must solve the system $Ax = e_i$ for each standard basis vector $e_i$. Instead of solving $[A|e_1], [A|e_2], \ldots, [A|e_n]$ individually, we can streamline this process by solving $[A|I]$ all at once.

If $A$ doesn't have $n$ pivots (one in each row), then $A^{-1}$ does not exist.

If $A$ has $n$ pivots, then the algorithm produces $A^{-1}$.

\subsection{Summary: Finding Inverses for $n \times n$ Matrices}

\begin{itemize}
    \item If $A$ doesn't have $n$ pivots, then $A^{-1}$ doesn't exist
    \item If $A$ has $n$ pivots, then the algorithm produces $A^{-1}$
\end{itemize}

\subsection{Example: Find $A^{-1}$, if it exists}

\[
A = \begin{bmatrix}
2 & 1 & 3 \\
2 & 6 & 8 \\
6 & 8 & 18
\end{bmatrix}
\]

Applying Gauss-Jordan elimination to $[A|I]$:

\[
\begin{bmatrix}
2 & 1 & 3 & | & 1 & 0 & 0 \\
2 & 6 & 8 & | & 0 & 1 & 0 \\
6 & 8 & 18 & | & 0 & 0 & 1
\end{bmatrix} \sim
\begin{bmatrix}
1 & 0 & 0 & | & 0 & 1 & -1/3 \\
0 & 1 & 0 & | & 1/2 & -1/4 & 1/4 \\
0 & 0 & 1 & | & -1/6 & 1/12 & 1/12
\end{bmatrix}
\]

Therefore,

\[
A^{-1} = \begin{bmatrix}
0 & 1/2 & -1/6 \\
1 & -1/4 & 1/12 \\
-1/3 & 1/4 & 1/12
\end{bmatrix}
\]

\subsection{Using Inverses to Solve Systems}

If $A$ exists, we can solve $Ax = b$ by multiplying both sides by $A^{-1}$:

\begin{align*}
A^{-1}Ax &= A^{-1}b \\
Ix &= A^{-1}b \\
x &= A^{-1}b
\end{align*}

Note: $Ax = b$ has solution $x = A^{-1}b$. However, even if $A$ has $n$ pivots (so it's invertible), this solution is unique.

\end{document}