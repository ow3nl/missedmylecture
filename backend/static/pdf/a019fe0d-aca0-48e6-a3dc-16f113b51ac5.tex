\documentclass[12pt,a4paper]{article}
\usepackage[margin=1in]{geometry}
\usepackage{amsmath,amssymb}

\begin{document}

\section{Sets}

\subsection{Basic Definitions}

$\mathbb{N} = \{0,1,2,\ldots\}$ \textit{natural numbers}

$\mathbb{Z} = \{\ldots,-1,0,1,\ldots\}$ \textit{integers}

$\mathbb{Q} = \{\frac{a}{b} : a,b \in \mathbb{Z}, b \neq 0\}$ \textit{rationals}

$\mathbb{R}$ - \textit{real numbers}

$\mathbb{C} = \{a+bi : a,b \in \mathbb{R}\}$ \textit{complex}

$\mathbb{R}^2$ - \textit{2-dimensional real plane}

$\mathbb{R}^n = \{(x_1,\ldots,x_n) : x_i \in \mathbb{R}\}$ \textit{n-dimensional space}

$\mathbb{R}^\infty = \{(x_1,x_2,\ldots) : x_i \in \mathbb{R}\}$ \textit{space of infinite sequences of reals}

\subsection{Set Operations}

$A \cup B = \{x : x \in A \text{ or } x \in B\}$ \textit{union}

$A \cap B = \{x : x \in A \text{ and } x \in B\}$ \textit{intersection}

$A \setminus B = \{x \in A : x \notin B\}$ \textit{set difference}

$A \triangle B = (A \setminus B) \cup (B \setminus A)$ \textit{symmetric difference}

\subsection{Examples}

Let $A_n = \{x_1,\ldots,x_n\}$ be an indexed family (where $n \in \mathbb{N}$).

\begin{align*}
\bigcup_{i=1}^n A_i &= \{x : x \in A_i \text{ for some } i \in \mathbb{N}\} \\
\bigcap_{i=1}^n A_i &= \{x : x \in A_i \text{ for every } i \in \mathbb{N}\}
\end{align*}

These are generalized to arbitrary indexed families.

\section{Functions}

A function $f: A \rightarrow B$ is a mapping that assigns to each $a \in A$ a unique $b \in B$.

$A_f = \{(a,f(a)) : a \in A\}$ is the \textit{graph} of $f$.

If $f: A \rightarrow B$ is a function, then:

\begin{itemize}
    \item $f$ is \textit{injective} (1-1) if $\forall a_1, a_2 \in A, f(a_1) = f(a_2) \Rightarrow a_1 = a_2$
    \item $f$ is \textit{surjective} (onto) if $\forall b \in B, \exists a \in A : f(a) = b$
    \item $f$ is \textit{bijective} if it is both injective and surjective
\end{itemize}

If $f: A \rightarrow B$ is bijective, then there exists a unique function $f^{-1}: B \rightarrow A$ called the \textit{inverse of f} (denoted by $f^{-1} = f_{inv}$).

\subsection{Examples of Functions}

\begin{enumerate}
    \item $f: \mathbb{R} \rightarrow \mathbb{R}, x \mapsto x^2$
    \item $f: \mathbb{R}^2 \rightarrow \mathbb{R}, (x,y) \mapsto x+y$
    \item $g: \mathbb{R} \rightarrow \mathbb{R}, x \mapsto e^x$
    \item $h: \mathbb{R} \rightarrow \mathbb{R}, x \mapsto \sin x$
    \item $f: \mathbb{C} \rightarrow \mathbb{C}, z \mapsto z^2 + 1$
\end{enumerate}

\end{document}